% NAS explanation and why it is important
In the wake of the deep learning revolution, deep learning is being applied to perform an increasingly diverse set of tasks. Applying deep learning traditionally involves a team of deep learning and domain experts that tailor a neural network architecture towards a specific task via a lengthy process of trial and error. To automate this process, researchers invented \textit{Neural Architecture Search} (NAS)~\cite{nas_survey_2019} methods. NAS methods employ diverse strategies to automatically search for a neural network architecture for a given deep learning task within an architecture search space.

% FL explanation and why it is important
Independently, but in parallel to NAS, researchers developed a machine learning approach called \textit{Federated Learning} (FL)~\cite{fl_seminal_2017} which enables training a ML model on data that is distributed across a set of \textit{clients} without sharing their data. The model is trained in \textit{communication rounds} by clients using their local data, wherafter model parameter updates are sent to a \textit{server} for aggregation and re-dispersal for the next round. FL enhances the privacy of client data and allows training even when clients' data is too large to be transferred effectively to a central training site or when clients' data can not be shared due to regulatory or privacy reasons.

% why NAS for FL is useful: NAS can deal with heterogenity 
When practitioners\footnote{In this thesis, \textit{practitioners} refers to both users and developers of FL and FedNAS methods.} make use of deep learning in FL, they typically use a predefined architecture. However, clients are heterogenous in both their data distributions and hardware capabilities, and since practitioners can not view either of these, picking a single architecture that will generalise well on each client's data and result in a model with acceptable inference latency on each client is difficult and time-consuming. Using NAS methods in FL offers an alternative approach. An architecture search can be coordinated across clients and be guided by performance feedback from candidate architectures evaluated on each client — allowing the NAS method to effectively see clients' data and hardware capabilities. Thus NAS can be used in FL to automatically find an architecture that generalises well~\cite{fednas_2021} and has acceptable inference latency on all clients~\cite{decnas_2022}, despite the inherent heterogeneity and invisibility of client characteristics.

% overall problem in "adapting NAS to FL" domain
Despite the benefits, using NAS in FL is not straightforward. Most NAS methods were developed with a centralised setting in mind and work due to assumptions that can be made in a centralised setting. In FL, many of the centralised assumptions do not hold, creating several \textit{challenges} for using NAS in FL and practitioners need to \textit{adapt} the NAS method or the FL pipeline to overcome them. % TODO: => FedNAS method is adaptation, add footnote explaining why this is the case
For example, in a centralised setting, practitioners can expect a NAS method to run on worker nodes connected via low-latency, high-bandwidth links. In FL however, a NAS method potentially runs on clients connected via an unreliable, high-latency and low-bandwith network, resulting in a challenge when transferring large amounts of architecture weights between clients and the server for coordinating an architecture search. Practitioners have developed various techniques to overcome this specific challenge, like compressing the architecture[TODO: cite] before transferring it or only transferring relevant subnetworks of an architecture~\cite{fedoras_2022}.

% practical problem: hard to re-use knowledge on adapting NAS to FL effectively
When practitioners adapt a centralised NAS method to FL, they typically target specific FL system characteristics. The relevance of a challenge for that FedNAS method depend on the extent to which the targeted FL system characteristics violate centralised NAS assumptions. Practitioners have already developed a plethora of FedNAS methods, but NAS is a rapidly evolving field of research and practitioners are constantly adapting new NAS methods to FL or adapt already adapted centralised NAS methods to a new set of FL system characteristics. Developing adaptation techniques to overcome challenges involves time-consuming iterations of testing and adjusting the techniques. Therefore, practitioners stand to benefit from comparing, re-using or building on top of tried and tested adaptation techniques. 

To this end practitioners need to comb the increasingly large body of FedNAS literature for suitable techniques for their targeted FL system charecteristics, but doing so is difficult. They need to weigh which adaptation techniques to use to overcome their relevant set of challenges, since an adaptation technique for overcoming one challenge may inadvertantly make it harder to overcome another. 

% # Gap
% ## "Situate in research: What are the conclusions of existing literature for that problem / situation in this domain?"
% ## "What is the problem or issue with that existing literature?"
Prior surveys of the FedNAS literature focus on the FedNAS method as a whole instead of the individual adaptation techniques a FedNAS method uses. Additionally, they only analyse a small fraction of the FedNAS literature. \cite{fl_to_nas_survey_2021} categorises FedNAS methods into offline versus online architecture search and single-objective versus multi-objective methods. \cite{nas_hpo_fl_survey_2023} gives a brief overview of the FedNAS landscape at the time as part of a larger survey into combining NAS and Hyperparameter Optimisation. It highlights the major scientific contributions each FedNAS method has made. \cite{multi-objective_methods_in_fl_2025} provides an overview of how multi-objective optimisation can be integrated into FL in general and includes sections that discuss how this is done specifically for FedNAS methods. Unfortunately, knowledge on potential challenges that occur when adapting a centralised NAS method to FL remains fragmented and an extensive overview of how adaptation techniques overcome them is lacking, leading to our research question:

\vspace{1em}
\textit{What challenges arise when adapting centralised NAS methods to FL and how do adaptation techniques in the literature overcome these challenges?}
\vspace{1em}

% # "Study: How investigate? Process, context & why?"
% ## "Indicate that this study addresses that problem or issue and state how."
% ## "Describe the study, sample, and method for addressing that problem or issue."
To tackle our research question, we conduct a systematic literature review of literature that presents FedNAS methods which modify centralised NAS methods in response to the FL setting.

We conduct a literature search using Scopus~\cite{scopus} to collect relevant publications. We include FedNAS literature that explicitly modifies a centralised NAS method for FL, and exclude FedNAS methods built from scratch, concern only hyperparameter optimisation, or methods with insufficient detail. We then recursively add eligible literature from references.

We analyse the literature through the lens of adaptation technique. We extract adaptation techniques through open coding and iteratively refine them afterwards. We map each technique's effect onto each challenge and summarise results in a challenge-technique table for practitioners.

% # "Conclusion"
% ## "Describe what you found. State explicitly how these findings extend and contribute to existing knowledge."
Our review aims to support the adaptation of centralised NAS methods to FL. To this end, we make the following contributions. By providing an overview of the challenges arising from discrepancies in assumptions between centralised NAS and FedNAS, we make explicit what troubles practitioners can expect to run into based on past experiences. [TODO: connect with previous sentence] By describing how FL system parameters influence the relevance of challenges, we enable practitioners to focus on the challenges relevant to their targeted FL system parameters. [TODO: connect with previous sentence] By extracting adaptation techniques and highlighting the challenges they address, we enable practitioners to reuse existing adaptation techniques for adapting centralised NAS methods to FL for their specific set of challenges.

% The outline paragraph (at `./03_outline.tex`) isn't included here, as it is not required by the exposé (it would be redundant due to the "Structure" section in the exposé). 
% It is included in the thesis document via `../../main.tex`.