% TODO at the end: make the "adaptation technique" lens + challenges mapping part of the solution

% NAS explanation and why it is important
In the wake of the deep learning revolution, deep learning is being applied to perform an increasingly diverse set of tasks. Applying deep learning traditionally involves a team of deep learning and domain experts that tailor a neural network architecture towards a specific task via a lengthy process of trial and error. To automate this process, researchers invented \textit{Neural Architecture Search} (NAS)~\cite{nas_survey_2019} methods and improved them over the past decade. NAS methods employ diverse strategies to automatically search for a neural network architecture for a given deep learning task within an architecture search space and allow optimising the architecture w.r.t. .

% FL explanation and why it is important
Independently, but in parallel to NAS, researchers developed a machine learning approach called \textit{Federated Learning} (FL)~\cite{fl_seminal_2017} which enables training a ML model on data that is distributed across a set of \textit{clients} and enhances the privacy of that data. Training in FL is coordinated by a \textit{server} that aggregates model updates computed by the clients locally into a globally shared model.

% why NAS for FL is useful: NAS can deal with heterogenity 
When practitioners\footnote{In this thesis, \textit{practitioners} refers to both users and developers of FL and FedNAS methods.} make use of deep learning in FL, they typically use a predefined architecture. However, clients' are heterogenous in both their data distributions and hardware capabilities in FL, and since practitioners can not view either of these, picking a single architecture that will generalise well on each client's data and result in a model with acceptable inference latency on each client surmounts to guesswork. Using NAS methods in FL offers an alternative approach. An architecture search can be coordinated across clients and be guided by performance feedback from candidate architectures evaluated on each client — allowing the NAS method to effectively see clients' data and hardware capabilities. Thus with NAS in FL, an architecture can be automatically found that generalises well~\cite{fednas_2021} and has acceptable inference latency on all clients~\cite{decnas_2022}.

% overall problem in "adapting NAS to FL" domain
Despite the benefits, using NAS in FL is not straightforward. Most NAS methods were developed with a centralised setting in mind and work due to assumptions that can be made in a centralised setting. In FL, these assumptions do not hold, requiring NAS methods to be \textit{adapted} to the FL setting. These assumption discrepancies result in \textit{challenges} for adapting NAS methods to FL, and practitioners have created various techniques for overcoming them, which we shall call \textit{adaptation techniques}. For example, in a centralised setting, practitioners can expect a NAS method to run on worker nodes connected via low-latency, high-bandwidth links. In FL however, a NAS method potentially runs on clients connected via an unreliable, high-latency and low-bandwith network. This is a challenge for supernet-based NAS methods in particular, since the server would need to transfer the updated weights of the entire supernet to each client after aggregation, which would take so long as to make it unworkable in practice. One adaptation technique overcomes this challenge by only sending a randomly sampled subspace of the supernet such that the size of the sent weights fits within a budget that can realistically be expected to take a short amount of time to transfer over the network.

% core problem: can't do fine-grained re-use and compare from the literature
%Pracitioners consntaly adapt new NAS methods to FL.
%- developing FedNAS method is expensive
%- want to re-use knowledge gained by prior adaptations of NAS to FL
%- have been past literature reviews, but all focus on the whole end-result thing
%- missing knowledge: the NAS methods are adapted and fine-grained, re-usable technique

% practical problem
NAS is a rapidly evolving field of research and practitioners are constantly adapting new NAS methods to FL to create potentially better FedNAS methods, but developing adaptation techniques to overcome challenges involves time-consuming iterations of testing and adjusting the techniques. Additionally, adapting a NAS method to FL involves trading off the effects of adaptations techniques on challenges. An adaptation may overcome one challenge, but inadvertantly make it harder to overcome another. Also, since implementing an adaptation technique takes signficant time and testing, it is in the interest of practitioners to implement only adaptation techniques that overcome the set of challenges relevant for their targeted FL system characteristics. For example, if clients can be expected to posess high-end hardware, challenges caused by low-end hardware in FL become less relevant.Therefore practitioners stand to benefit from comparing, re-using or building on top of tried and tested adaptation techniques. However, finding potentially useful adaptation techniques requires practitioners to comb the increasingly large body of FedNAS literature for techniques and evaluate whether they are a good fit.

% # Gap
% ## "Situate in research: What are the conclusions of existing literature for that problem / situation in this domain?"
% ## "What is the problem or issue with that existing literature?"
% was machen die surveys genau und wie grenzt sich das von deinem Vorhaben ab?
A single FedNAS method can make use of multiple adaptation techniques, each of which could be independantly re-used and may overcome a different challenge, but prior surveys of the FedNAS literature focus on the FedNAS method as a whole instead. Additionally, they only analyse a small fraction of the FedNAS literature. \cite{fl_to_nas_survey_2021} categorises FedNAS methods into offline versus online architecture search and single-objective versus multi-objective methods. \cite{nas_hpo_fl_survey_2023} gives a brief overview of the FedNAS landscape at the time as part of a larger survey into combining NAS and Hyperparameter Optimisation. It highlights the major scientific contributions each FedNAS method has made. \cite{multi-objective_methods_in_fl_2025} provides an overview of how multi-objective optimisation can be integrated into FL in general and includes sections that discuss how this is done specifically for FedNAS methods.
There is a lack of an overview of the potentially relevant challenges that practitioners face when adapting NAS methods to FL as well as an extensive overview of adaptation techniques used by FedNAS methods to overcome those challenges, which leads to our research question: 

\vspace{1em}
\textit{What challenges arise when adapting NAS methods developed in a centralised setting to FL and how do adaptation techniques in the literature overcome these challenges?}
\vspace{1em}
% Alternative RQ, that doesn't prematurely introduce the concept of an adaptation technique: 
% What challenges arise when adapting NAS methods developed in a centralised setting to FL and how are they overcome in the literature?
% Maybe the fact that there is a disitinction between a centralised vs. decentralised NAS method should also be part of the solution and not part of the research question

% # "Study: How investigate? Process, context & why?"
% ## "Indicate that this study addresses that problem or issue and state how."
% ## "Describe the study, sample, and method for addressing that problem or issue."
To tackle our research question, we conduct a systematic literature review of literature that presents FedNAS methods which modify centralised NAS methods in response to the FL setting.

% ### from targeted FL settings and their challenges to adaptation techniques
We conduct a literature search using Scopus~\cite{scopus} to collect relevant publications. We include FedNAS literature that explicitly modifies a centralised NAS method for FL, and exclude FedNAS methods built from scratch, concern only hyperparameter optimisation, or methods with insufficient detail. We then recursively add eligible literature from references.

We analyse the literature through the lens of adaptation technique. We extract adaptation techniques through open coding and iteratively refine them afterwards. We map each technique's effect onto each challenge and summarise results in a challenge-technique table for practitioners.

% # "Conclusion"
% ## "Describe what you found. State explicitly how these findings extend and contribute to existing knowledge."
Our review aims to support the adaptation of centralised NAS methods to FL. To this end, we make the following contributions. By providing an overview of the challenges arising from discrepancies in assumptions between centralised NAS and FedNAS, we make explicit what troubles practitioners can expect to run into based on past experiences. [TODO: connect with previous sentence] By describing how FL system parameters influence the relevance of challenges, we enable practitioners to focus on the challenges relevant to their targeted FL system parameters. [TODO: connect with previous sentence] By extracting adaptation techniques and highlighting the challenges they address, we enable practitioners to reuse existing adaptation techniques for adapting centralised NAS methods to FL for their specific set of challenges.

% The outline paragraph (at `./03_outline.tex`) isn't included here, as it is not required by the exposé (it would be redundant due to the "Structure" section in the exposé). 
% It is included in the thesis document via `../../main.tex`.