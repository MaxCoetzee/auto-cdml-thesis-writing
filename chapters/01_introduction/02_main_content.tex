% TODO at the end: make the "adaptation technique" lens + challenges mapping part of the solution

% NAS explanation and why it is important
In the wake of the deep learning revolution, deep learning is being applied to perform an increasingly diverse set of tasks. Applying deep learning traditionally involves a team of deep learning and domain experts tailoring a neural network architecture towards the task at hand via a lengthy process of trial and error. To automate this process, researchers invented \textit{Neural Architecture Search} (NAS)~\cite{nas_survey_2019} methods and improved them over the past decade. NAS methods employ diverse strategies to automatically search for a neural network architecture for a given deep learning task — typically by iteratively evaluating the performance of a set of candidate architectures and selecting the best ones [TODO: confirm].

% FL explanation and why it is important
Independently, but in parallel to NAS, researchers developed a machine learning approach called \textit{Federated Learning} (FL)~\cite{fl_seminal_2017} which enables training a ML model on data that is distributed across a set of \textit{clients} and enhances the privacy of that data. Training in FL is coordinated by a \textit{server} that aggregates model updates computed by the clients locally into a globally shared model.

% why NAS for FL is useful: NAS can deal with client data heterogenity 
Practitioners\footnote{In this thesis, \textit{practitioners} refers to both users and developers of FL and FedNAS methods.} making use of deep learning in FL typically [TODO: reconciliate with supervisor comment] select a neural network architecture upfront — a so-called \textit{predefined architecture}. However, because practitioners can not see clients' data and the data distributions vary significantly in FL [TODO: contrast with typical ML to show why this is a problem?], they may select an architecture which does not generalise well on some of the client data distributions. This leads to a slow, unstable training process and [TODO: only makes this difficult] likely results in a model with low accuracy and, on less powerful clients [TODO: because of the one size fits all approach], results in high inference latency. Practitioners have started using NAS methods instead of predefined architectures in FL to mitigate these issues. NAS methods can search for architectures suited to the clients' data guided by performance feedback from candidate architectures trained on clients. As a result, the searched architecture can potentially speed up training convergence~\cite{fed_meta_nas_2025}, make training more stable~\cite{hapfnas_2025}, increase the accuracy of the globally aggregated model~\cite{fednas_2021} and reduce the trained model's inference latency~\cite{decnas_2022} — all depending on the optimisation objectives and constraints of the NAS method.

% overall problem in "adapting NAS to FL" domain
Despite the benefits, using NAS in FL is not straightforward. Most NAS methods were developed with a centralised setting in mind and work due to assumptions that can be made in a centralised setting. In FL, these assumptions do not hold, requiring most NAS methods to be \textit{adapted} to the FL setting. These assumption discrepancies result in \textit{challenges} for adapting centralised NAS methods to FL, and practitioners have created various techniques for overcoming challenges, which we shall call \textit{adaptation techniques}. [TODO: reword clients vs. worker nodes clarity] For example, centralised NAS can assume worker nodes are connected via low-latency, high-bandwidth links, but clients in FL are not. [TODO: more concise, less convoluted] This poses a problem for using unmodified supernet-based NAS methods in particular, since the server would need to transfer the updated weights of the entire supernet to each client after aggregation, which would take so long as to make it unworkable in practice. One adaptation technique overcomes this challenge by only sending a randomly sampled subspace of the supernet such that the size of the sent weights fits within a budget that can realistically be expected to take a short amount of time to transfer over the network.

% part of pracitcal problem?: the relevant challenge subset depends on the targeted FL setting, which varies a lot
The relevant subset of challenges practitioners face, when adapting a centralised NAS method to FL depends on the characteristics of the targeted FL system~\cite{fl_advances_and_open_problems_2021}. For example, if clients can be expected to posess high-end hardware, challenges caused by low-end hardware in FL become less relevant. [TODO: should list other characteristics?] Since implementing an adaptation techniques takes signficant time and testing, it is in the interest of practitioners to implement only adaptation techniques that overcome the set of challenges relevant for their targeted FL system characteristics. Additionally some techniques may worsen some challenges while overcoming others.

% practical problem
% TODO: add point? "adapting NAS methods to an as-yet untargeted set of FL system characteristics" 
[TODO: rephrase sentence] The field of NAS research is moving fast and practitioners are constantly adapting new NAS methods to FL, but doing so is difficult. Developing adaptation techniques involves time-consuming iterations of testing and adjusting the techniques on expensive compute setups. Therefore practitioners benefit from re-using or building on top of tried and tested adaptation techniques. However, finding potentially useful adaptation techniques requires practitioners to comb the increasingly large body of FedNAS literature for techniques and evaluate whether they are a good fit.

Practitioners adapting a centralised NAS method to FL with a specific set of FL system characteristics in mind must decide which challenges to prioritise and which adaptation techniques to implement. However, the literature does not offer clear guidance for these design decisions. The incurred challenges are scattered throughout the literature, the adaptation techniques used to address them are often not presented in isolation, and a large portion of the FedNAS literature does not clearly state the targeted FL system parameters. As a result, practitioners struggle to assess the usefulness of existing adaptation techniques for their targeted FL system parameters and risk selecting ineffective techniques or selecting techniques for addressing a challenge that are known to worsen another.

% # Gap
% ## "Situate in research: What are the conclusions of existing literature for that problem / situation in this domain?"
% ## "What is the problem or issue with that existing literature?"
% was machen die surveys genau und wie grenzt sich das von deinem Vorhaben ab?
A single FedNAS method can make use of multiple adaptation techniques, each of which could be independantly re-used and may overcome a different challenge, but prior surveys of the FedNAS literature focus on the FedNAS method as a whole instead. Additionally, they only analyse a small fraction of the FedNAS literature. \cite{fl_to_nas_survey_2021} categorises FedNAS methods into offline versus online architecture search and single-objective versus multi-objective methods. \cite{nas_hpo_fl_survey_2023} gives a brief overview of the FedNAS landscape at the time as part of a larger survey into combining NAS and Hyperparameter Optimisation. It highlights the major scientific contributions each FedNAS method has made. \cite{multi-objective_methods_in_fl_2025} provides an overview of how multi-objective optimisation can be integrated into FL in general and includes sections that discuss how this is done specifically for FedNAS methods.
There is a lack of an overview of the potentially relevant challenges that practitioners face when adapting NAS methods to FL as well as an extensive overview of adaptation techniques used by FedNAS methods to overcome those challenges, which leads to our research question: 

\vspace{1em}
\textit{What challenges arise when adapting NAS methods developed in a centralised setting to FL and how do adaptation techniques in the literature overcome these challenges?}
\vspace{1em}

% # "Study: How investigate? Process, context & why?"
% ## "Indicate that this study addresses that problem or issue and state how."
% ## "Describe the study, sample, and method for addressing that problem or issue."
To tackle our research question, we conduct a systematic literature review of literature that presents FedNAS methods which modify centralised NAS methods in response to the FL setting.

% ### from targeted FL settings and their challenges to adaptation techniques
We conduct a literature search using Scopus~\cite{scopus} to collect relevant publications. We include FedNAS literature that explicitly modifies a centralised NAS method for FL, and exclude FedNAS methods built from scratch, concern only hyperparameter optimisation, or methods with insufficient detail. We then recursively add eligible literature from references.

We extract adaptation techniques through open coding and iteratively refine them afterwards. We map each technique's effect onto each challenge and summarise results in a challenge-technique table for practitioners.

% # "Conclusion"
% ## "Describe what you found. State explicitly how these findings extend and contribute to existing knowledge."
Our review aims to support the adaptation of centralised NAS methods to FL. To this end, we make the following contributions. By providing an overview of the challenges arising from discrepancies in assumptions between centralised NAS and FedNAS, we make explicit what troubles practitioners can expect to run into based on past experiences. [TODO: connect with previous sentence] By describing how FL system parameters influence the relevance of challenges, we enable practitioners to focus on the challenges relevant to their targeted FL system parameters. [TODO: connect with previous sentence] By extracting adaptation techniques and highlighting the challenges they address, we enable practitioners to reuse existing adaptation techniques for adapting centralised NAS methods to FL for their specific set of challenges.

% The outline paragraph (at `./03_outline.tex`) isn't included here, as it is not required by the exposé (it would be redundant due to the "Structure" section in the exposé). 
% It is included in the thesis document via `../../main.tex`.