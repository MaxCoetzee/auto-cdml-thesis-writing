% NAS explanation and why it is important
Engineering the architecture of a neural network for a Deep Learning application is traditionally done by a team of experts via a process of trial and error. To reduce the amount of manual labour involved in this process, researchers invented \textit{Neural Architecture Search} (NAS)~\cite{nas_survey_2019} methods and improved them over the past decade. NAS methods employ diverse strategies to automatically search for a neural network architecture for a given Deep Learning application.

% FL explanation and why it is important
Independently, but in parallel to NAS, researchers developed a distributed machine learning approach called \textit{Federated Learning} (FL)~\cite{fl_seminal_2017} in response to growing concerns about data privacy. In FL, \textit{clients} collaboratively train a model without sharing their local data. This enhances the privacy of clients' data by ensuring that FL model trainers cannot view clients' data and that client data is not collected at a central location, where a single breach could expose the data of all clients.

% why NAS for FL is useful: reduces manual labour and, specific to FL, predefined architectures do not work well in FL 
Engineering neural network architectures in FL is as labour-intensive as in centralised Deep Learning, therefore practitioners\footnote{In this thesis, \textit{practitioners} referes to both users and developers of FedNAS methods.} have started investigating the use of NAS methods in FL~\cite{fednas_2021}~\cite{fedoras_2022}~\cite{finch_2024}, creating \textit{Federated Neural Architecture Search methods} (FedNAS methods)~\cite{fednas_2021}. FedNAS methods also provide a potential alternative to selecting a fixed architecture upfront — a so-called \textit{predefined architecture}. Predefined architectures can lead to slow training convergence and poorly performing models in FL, because practitioners cannot view the clients' data, and client hardware capabilities vary. Practitioners may therefore select a predefined architecture that contains components irrelevant for generalising well from client data sets or select an architecture that trains slowly on some clients. Work has already been done that shows the use of NAS methods can mitigate these issues~\cite{fedpnas_2021}~\cite{spider_2021}~\cite{peaches_2024}.

% overall problem in "adapting NAS to FL" domain
The approach used by most FedNAS methods — and on which we focus in this thesis — is to take NAS methods developed in a centralised setting and use them for FL. However, this requires modifying the NAS method for the FL setting, because many assumptions that hold for the search process in centralised NAS do not hold for FedNAS. These assumption discrepancies result in \textit{challenges} for using centralised NAS methods in FL, and practitioners have created \textit{adaptation techniques} for overcoming them. For example, the FedNAS method \textit{FedorAS}~\cite{fedoras_2022} makes use of the centralised NAS method \textit{SPOS}~\cite{spos_2020}. Since centralised NAS can assume worker nodes are connected via low-latency, high-bandwidth links, but clients in FL are not, the following challenge arises: sending the parameters of the entire supernet used in SPOS to each client would take an infeasible amount of time. Instead, \textit{FedorAS} adapts \textit{SPOS} such that only a small, sampled subspace of the supernet is sent to each client for training and evaluation.

% the relevant challenge subset depends on the targeted FL setting, which varies a lot
The relevant subset of challenges faced by these kinds of FedNAS methods depends on the configuration of FL system parameters~\cite{fl_advances_and_open_problems_2021}. System parameters include the average hardware capabilities of clients, the average network latency of clients, the number of participating clients, etc. For example, in the so-called \textit{cross-silo setting}, clients can be expected to be equipped with GPUs, making challenges caused by low-end hardware in FL less relevant.

% practical problem: hard to make design decisions based on current literature
Practitioners adapting a centralised NAS method to FL with a specific set of FL system parameters in mind must decide which challenges to prioritise and which adaptation techniques to implement. However, the literature does not offer clear guidance for these design decisions. The incurred challenges are scattered throughout the literature, the adaptation techniques used to address them are often not presented in isolation, and a large portion of the FedNAS literature does not clearly state the targeted FL system parameters. As a result, practitioners struggle to assess the usefulness of existing adaptation techniques for their targeted FL system parameters and risk selecting ineffective techniques or selecting techniques for addressing a challenge that are known to worsen another.

% # Gap
% ## "Situate in research: What are the conclusions of existing literature for that problem / situation in this domain?"
% ## "What is the problem or issue with that existing literature?" (practical problem)
Prior literature surveys~\cite{fl_to_nas_survey_2021}~\cite{nas_hpo_fl_survey_2023}~\cite{multi-objective_methods_in_fl_2025} only summarise FedNAS methods on the whole and do not focus on isolated, re-usable adaptation techniques. Additionally, prior surveys are limited by the FedNAS methods available at the time or exclude a large share of the FedNAS literature due to their chosen focus. As a result, prior surveys do not provide an exhaustive overview of adaptation techniques and do not help practitioners assess the usefulness of existing adaptation techniques for their targeted FL system parameters. Since prior surveys do not solve the problem mentioned in the previous paragraph, we pose the following research question to help with the adaptation of NAS methods to FL:

\vspace{1em}
% TODO: unsatisfactory and long
\textbf{What challenges arise when adapting centralised NAS methods to FL? How is the relevance of challenges influenced by FL system parameters, and how do adaptation techniques in the literature overcome these challenges?}
\vspace{1em}

% # "Study: How investigate? Process, context & why?"
% ## "Indicate that this study addresses that problem or issue and state how."
% ## "Describe the study, sample, and method for addressing that problem or issue."
To tackle our research question, we conduct a systematic literature review of literature that presents FedNAS methods which modify centralised NAS methods in response to the FL setting.

% ### from targeted FL settings and their challenges to adaptation techniques
We collect relevant literature by searching for titles, abstracts and keywords in Scopus~\cite{scopus} using the search string "federated learning neural architecture search". We include FedNAS literature that explicitly modifies a centralised NAS method for FL, and exclude FedNAS methods built from scratch, concern only hyperparameter optimisation, or methods with insufficient detail. We then recursively add eligible literature from references.

We conduct a systematic qualitative literature review, building a conceptual model linking violated centralised NAS assumptions to FL challenges, the effect of FL system parameters on challenge relevance, and adaptation techniques. We extract adaptation techniques through open coding and iteratively refine them afterwards. We map each technique's effect onto each challenge and summarise results in a challenge-technique table for practitioners.

% # "Conclusion"
% ## "Describe what you found. State explicitly how these findings extend and contribute to existing knowledge."
Our review aims to support the adaptation of centralised NAS methods to FL. To this end, we make the following contributions:
\begin{enumerate}
    \item We help practitioners understand the challenges arising from discrepancies in assumptions between centralised NAS and FedNAS.
    \item We help practitioners understand which challenges are relevant for their targeted FL system parameters.
    \item By extracting adaptation techniques and highlighting the challenges they address, we enable practitioners to reuse existing adaptation techniques for adapting centralised NAS methods to FL for their specific set of challenges.
\end{enumerate}

% The outline paragraph (at `./03_outline.tex`) isn't included here, as it is not required by the exposé (it would be redundant due to the "Structure" section in the exposé). 
% It is included in the thesis document via `../../main.tex`.