% NAS explanation and why it is important
Engineering the architecture of a neural network for a Deep Learning application is traditionally done by a team experts via a process of trial and error. To reduce the amount of manual labour involved in this process, researchers invented \textit{Neural Architecture Search} (NAS)~\cite{nas_survey_2019} methods and improved them over the past decade. NAS methods employ diverse strategies to automatically search for a neural network architecture for a given Deep Learning application.

% FL explanation and why it is important
Independantly, but in parallel to NAS, researchers developed a distributed machine learning approach called \textit{Federated Learning} (FL)~\cite{fl_seminal_2017} in response to growing concerns about data privacy. In FL, \textit{clients} collaboratively train a model without sharing their local data. This enhances the privacy of clients' data by ensuring FL model trainers can not view clients' data and client data is not collected at a central location where a single breach could expose the data of all clients.

% why NAS for FL is useful: reduces manual labour and, specific to FL, predefined architectures do not work well in FL 
Engineering neural network architectures in FL is as labour-intensive as in centralised Deep Learning, therefore practitioners\footnote{In this thesis, \textit{practitioners} referes to both users and developers of FedNAS methods.} have started investigating the use of NAS methods in FL~\cite{fednas_2021}~\cite{fedoras_2022}~\cite{finch_2024}, creating \textit{Federated Neural Architecture Search methods} (FedNAS methods)~\cite{fednas_2021}. FedNAS methods also provide a potential alternative to selecting a fixed architecture upfront — a so-called \textit{prededfined architecture}. Predefined architectures can lead to slow training convergence and poorly performing models in FL, because practitioners can not view the clients' data and client hardware capabilities vary. Practitioners may therefore select a predefined architecture that contains components irrelevant for generalising well from client data sets or select an architecture that trains slowly on some clients. Work has already been done that shows the use of NAS methods can mitigate these issues~\cite{fedpnas_2021}~\cite{spider_2021}~\cite{peaches_2024}.

% overall problem in "adapting NAS to FL" domain
The approach used by most FedNAS methods — and on which we focus in this thesis — is to take NAS methods developed in a centralised setting and use them for FL. However, this requires modifying the NAS method for the FL setting, because many assumptions that hold for the search process in centralised NAS do not hold for FedNAS. These assumption discrepancies result in \textit{challenges} for using centralised NAS methods in FL, and practitioners have created \textit{adaptation techniques} for overcoming them. For example, the FedNAS method \textit{FedorAS}~\cite{fedoras_2022} makes use of the centralised NAS method \textit{SPOS}~\cite{spos_2020}, but runs into the following challenge caused by the fact that centralised NAS can assume worker nodes are connected via low latency, high-bandwidth links, whereas clients in FL are not: sending the parameters of the entire supernet used in SPOS to each client would take an infeasible amount of time. Instead \textit{FedorAS} adapts \textit{SPOS} such that only a small, sampled subspace of the supernet is sent to each client for training and evaluation.

% the relevant challenge subset depends on the targeted FL setting, which varies a lot
The relevant subset of challenges faced by these kinds of FedNAS methods depends on the configuration of FL system parameters~\cite{fl_advances_and_open_problems_2021}. System parameters include average hardware capabilities of clients, average network latency of clients, the number of participating clients etc. For example, in the so-called \textit{cross-silo setting}, clients can be expected to be equipped with GPUs, making challenges caused by low-end hardware in FL less relevant.

% practical problem: hard to make design decisions based on current literature
Practititoners adapting a centralised NAS method to FL with a specific set of FL system parameters in mind must decide which challenges to prioritise and which adaptation techniques to implement, but the literature does not offer clear advice for these design decisions. The incurred challenges are scattered throughout the literature, the adaptation techniques used to address them are often not presented in isolation and many FedNAS papers do not clearly state the targeted FL system parameters. As a result, practitioners struggle to assess the usefuleness of existing adaptation techniques for their targeted FL system parameters and risk selecting ineffective techniques or selecting techniques for addressing a challenge that are known to worsen another.

% # Gap
% ## "Situate in research: What are the conclusions of existing literature for that problem / situation in this domain?"
% ## "What is the problem or issue with that existing literature?" (practical problem)
Prior literature surveys~\cite{fl_to_nas_survey_2021}~\cite{nas_hpo_fl_survey_2023}~\cite{multi-objective_methods_in_fl_2025} only summarise FedNAS methods on the whole and do not focus on isolated, re-usable adaptation techniques. Additionally, prior surveys are limited by the FedNAS methods available at the time or exclude a large share of the FedNAS literature due to their chosen focus. As a result, prior surveys don't provide an exhaustive overview of adaptation techniques and do not help practitioners assess the usefuleness of existing adaptation techniques for their targeted FL system parameters. Since prior surveys do not solve the problem mentioned in the previous paragraph, we pose the following research question help with the adpatation of NAS methods to FL:

\vspace{1em}
% TODO: unsatisfactory and long
\textbf{What challenges arise when adapting centralised NAS methods to FL, how is the relevance of challenges influenced by FL system parameters and how do adaptation techniques in the literature overcome these challenges?}
\vspace{1em}

% # "Study: How investigate? Process, context & why?"
% ## "Indicate that this study addresses that problem or issue and state how."
% ## "Describe the study, sample, and method for addressing that problem or issue."
To tackle our research question, we conduct a systematic literature review of papers that present FedNAS methods which modify centralised NAS methods in response to the FL setting.

% ### from targeted FL settings and their challenges to adaptation techniques
% TODO: "was du machst ist gut, aber wie genau ist unklar. Wie willst du die paper analysieren um diese Elemente zu extrahieren? Thematic analysis oder ein anderes coding Verfahren?"
We define a set of fine-grained parameters to characterise the targeted FL setting of each FedNAS method based on observations of varying setting parameters in the literature. 

With the help of this characterisation, we identify the violated centralised NAS assumptions and catalogue the challenges that arise from them. Next, we extract unrefined adaptation techniques from the FedNAS methods and iteratively refine and merge them (similar to \cite{cdml_2024}) to obtain a set of collectively exhaustive adaptation techniques. We analyse how each adaptation technique works towards, against, or does not affect each challenge, and present our findings in the form of a discussion for each adaptation technique, as well as an overview table. 

% # "Conclusion"
% ## "Describe what you found. State explicitly how these findings extend and contribute to existing knowledge."
% - first sentence: what is the main goal
% - following sentences: contributions to knowledge and why they are relevant to research question
% TODO: "Häufig nummeriert man aber über die contributions, damit klar wird wie viele und welche man hat."
% TODO: "Satz 3 würde ich auch anders formulieren, das klarherauskommt wir haben adaptation techniques entwickelt zu den challenges und das ist jetzt für das gut."
Our review aims to support the adaptation of centralised NAS methods to FL. To this end we make the following contributions:
\begin{enumerate}
    \item We create an overview of challenges arising from assumption discrepancies between centralised NAS and FedNAS
\end{enumerate}
By identifying the source of challenges and elaborating on them, we provide clarity on the expected challenges for a targeted FL setting. Based on the expected challenges, FedNAS practitioners can use our overview of adaptation techniques to guide the design of new FedNAS methods and determine whether to re-use existing techniques, extend them, or develop new ones.

% The outline paragraph (at `./03_outline.tex`) isn't included here, as it is not required by the exposé (it would be redundant due to the "Structure" section in the exposé). 
% It is included in the thesis document via `../../main.tex`.