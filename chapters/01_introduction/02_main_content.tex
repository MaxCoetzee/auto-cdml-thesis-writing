% NAS explanation and why it is important
In the wake of the deep learning revolution, deep learning is being applied to an increasingly diverse set of tasks. Applying deep learning traditionally involves a team of deep learning and domain experts who tailor a neural network architecture to a specific task through a lengthy process of trial and error. To automate this process, researchers invented \textit{Neural Architecture Search} (NAS)~\cite{nas_survey_2019} methods. NAS methods employ diverse strategies to automatically search for a neural network architecture for a given deep learning task within an architecture search space.

% FL explanation and why it is important
Independently, but in parallel to NAS, researchers developed a machine learning approach called \textit{Federated Learning} (FL)~\cite{fl_seminal_2017}, which enables training an ML model on data distributed across a set of \textit{clients} without sharing their data. The model is disseminated from the server to clients and trained in \textit{communication rounds} by clients using their local data, after which model weight updates are sent to a \textit{server} for aggregation and re-dissemination in the next round. FL enhances the privacy of client data and enables training even when clients' data is too large to transfer effectively to a central training site or cannot be shared due to regulatory or privacy constraints.

% why NAS for FL is useful: NAS can deal with heterogeneity
When practitioners\footnote{In this thesis, \textit{practitioners} refers to both users and developers of FL and FedNAS methods.} make use of deep learning in FL, they typically use a predefined architecture. However, clients' data distributions and hardware capabilities are highly heterogeneous in FL. Since practitioners cannot directly observe these client characteristics~\cite{fednas_2021}, it is hard to choose a single architecture that generalises well across all clients and that meets inference latency constraints on every client. Using NAS methods in FL provides an alternative: the server can coordinate an architecture search across clients, guided by performance feedback from candidate architectures evaluated directly on clients — making clients' characteristics observable to the NAS method. Thus, the NAS method can better guide the architecture search than a practitioner could.

% overall problem in "adapting NAS to FL" domain
Despite the benefit, using NAS in FL is not straightforward. Most NAS methods were developed for a centralised setting and rely on assumptions that often do not hold for FL. These assumption discrepancies create several \textit{challenges} for using NAS in FL, and practitioners need to \textit{adapt} the NAS method or the FL pipeline to overcome them — creating a \textit{FedNAS}~\cite{fednas_2021} method. For example, in a centralised setting, practitioners can expect a NAS method to run on worker nodes connected via low-latency, high-bandwidth links. However, in FL, a NAS method may run on clients connected via an unreliable, high-latency, low-bandwidth network. This creates a challenge when transferring the weights of large candidate models between clients and the server to coordinate an architecture search. Practitioners have developed various \textit{adaptation techniques} to overcome this challenge, such as only transferring an encoding of a candidate architecture for clients to evaluate \cite{aging_fednas_2025}.

% practical problem: hard to re-use knowledge on adapting NAS to FL effectively
Practitioners stand to benefit from comparing and re-using existing adaptation techniques when creating FedNAS methods. To this end, practitioners need to comb the increasingly large body of FedNAS literature for suitable adaptation techniques — i.e. techniques that overcome the challenges relevant to their targeted FL system characteristics. The relevance of a challenge depends on the extent to which the targeted FL system characteristics violate centralised NAS assumptions. FL system characteristics include the average hardware capabilities of clients, the average network latency of clients, the number of participating clients, etc. Additionally, practitioners need to weigh which adaptation techniques to use to overcome their relevant set of challenges, since an adaptation technique for overcoming one challenge may inadvertently make it harder to overcome another.

% # Gap
% ## "Situate in research: What are the conclusions of existing literature for that problem / situation in this domain?"
% ## "What is the problem or issue with that existing literature?"
However, comparing and re-using adaptation techniques is difficult. Prior surveys of the FedNAS literature categorise FedNAS methods as a whole rather than analyse the individual adaptation techniques they use. \cite{fl_to_nas_survey_2021} categorises FedNAS methods into offline versus online architecture search and single-objective versus multi-objective methods. \cite{nas_hpo_fl_survey_2023} gives a high-level overview of the FedNAS landscape at the time of its publication as part of a larger survey into combining NAS and Hyperparameter Optimisation. \cite{multi-objective_methods_in_fl_2025} investigates how multi-objective optimisation can be integrated into FL in general and includes only parts that discuss how this is done specifically for FedNAS methods. All of the surveys analyse only a small fraction of the entire FedNAS literature. This means knowledge on adaptation techniques and potential challenges when adapting NAS methods to FL remains fragmented, leading to our research question:

\vspace{1em}
\textit{What challenges arise when adapting centralised NAS methods to FL, and how do adaptation techniques in the literature overcome these challenges?}
\vspace{1em}

% # "Study: How investigate? Process, context & why?"
% ## "Indicate that this study addresses that problem or issue and state how."
% ## "Describe the study, sample, and method for addressing that problem or issue."
To tackle our research question, we conduct an extensive systematic literature review. We collect an initial set of relevant literature with a Scopus~\cite{scopus} search and extend it with a snowball search. We first document the challenges that arise when adapting a centralised NAS method to FL, as well as the influence of FL system characteristics on the relevance of these challenges, in preparation for the next step. We then extract adaptation techniques from the literature through open coding and iteratively refine them to synthesise a coherent set of mutually exclusive and collectively exhaustive adaptation techniques. Next, we discuss how each adaptation technique works towards, against, or does not affect overcoming each challenge. Finally, we summarise our results in a table that allows practitioners to quickly find adaptation techniques to overcome a specific challenge.

% # "Conclusion"
% ## "Describe what you found. State explicitly how these findings extend and contribute to existing knowledge."
Our review aims to synthesise adaptation techniques from the literature to support their re-use and comparison in practice. We make the following three contributions. First, by providing a consolidated overview of the challenges of adapting centralised NAS methods to FL, practitioners can grasp which issues they can expect to encounter at a glance. Second, by describing how FL system characteristics influence the relevance of challenges, we enable practitioners to focus on the challenges relevant to them. Third, by extracting adaptation techniques and highlighting the challenges they address, we enable practitioners to compare existing adaptation techniques for adapting centralised NAS methods to FL for their specific set of challenges.

% The outline paragraph (at `./03_outline.tex`) isn't included here, as it is not required by the exposé (it would be redundant due to the "Structure" section in the exposé). 
% It is included in the thesis document via `../../main.tex`.