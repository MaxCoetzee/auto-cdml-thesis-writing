% !TeX root = ../main.tex
% Add the above to each chapter to make compiling the PDF easier in some editors.
\chapter{Introduction}\label{chapter:introduction}

% # "hook"
% ## hook a) "what is the domain and why is it important?"
Both \textit{Neural Architecture Search} (NAS) and \textit{Federated Learning} (FL) have made significant progress independently in the past decade. To benefit from the advantages of NAS methods in FL, researchers have started combining them by using NAS in FL~\cite{fednas_2021}~\cite{fedoras_2022}~\cite{finch_2024}. 
% => opening paragraph begs for
%      1. explanations of NAS and FL
%      2. why each is important
%      3. why applying NAS in FL is important 

% ### NAS explanation and why it is important
NAS automates the process of engineering neural network architectures for Deep Learning application domains~\cite{nas_survey_2019}. This saves manual labour compared to the traditional approach driven by a team of experts.

% ### FL explanation and why it is important
FL is a machine learning method whereby \textit{clients} collaboratively train a model without sharing their local data~\cite{fl_seminal_2017}. FL enables privacy-preserving machine learning on the increasing volume of privacy-sensitive, distributed data and avoids centrally collecting client data. 

% ## why NAS for FL is important: predefined architectures don't work well in FL 
Using NAS in FL reduces manual effort for practitioners and can be used to mitigate the issues with predefined architectures in FL. Since client data is invisible to practitioners and client data distributions as well as client hardware vary in FL, predefined architectures are often suboptimal. Some NAS methods can be used to tailor architectures to clients and improve FL training effectivenes~\cite{fedpnas_2021}~\cite{spider_2021}~\cite{peaches_2024}.

% ## hook b) "What is the overall problem or situation in that domain?"
% ### overall problem in FedNAS domain: assumption discrepancies create challenges for using NAS in FL
Despite the potential, NAS research has focused on the traditional centralised setting as opposed to the FL setting. This makes many NAS methods infeasible for direct application in FL, because centralised NAS can make several assumptions about the search process that do not hold in the FL setting. Each assumption discrepancy creates a \textit{challenge} for using NAS in FL and researchers have developed \textit{adaptation techniques} for overcoming them. Applying adaptation techniques to NAS methods results in \textit{Federated Neural Architecture Search} (FedNAS) methods~\cite{fednas_2021}.

% ### challenge example: Limited Computational Resources
For example, one challenge arises from the fact that centralised NAS can assume worker nodes are computationally powerful, whereas clients in most FL settings are not. Since most centralised NAS methods place large computational burdens on worker nodes, practitioners using these NAS methods in FL without modification would experience detrimental search completion times. Researchers have developed adaptation techniques for reducing the computational burden on individual clients in FedNAS methods~\cite{fedoras_2022}~\cite{efnas_2024}~\cite{nasfly_2024}. This typically involves reducing the overall computational work and splitting it up into smaller units, which results in a challenge, because the implementation of the resulting FedNAS method tends to be complex.

% ### the relevant challenge subset depends on the targeted FL setting, which varies a lot
The subset of challenges faced by FedNAS methods depend on the specific FL setting, which can differ in many aspects~\cite{fl_advances_and_open_problems_2021}. Each FL setting violates centralised NAS assumptions to a different extent, making some challenges more relevant than others for that setting. For example, two major classes of FL settings appear in the literature: the \textit{cross-device} class, wherein clients are edge devices, and the \textit{cross-silo} class wherein clients are entire organisations. For FL settings in the cross-silo class, clients can be expected to be equipped with GPUs, making the challenge described above less relevant.

% # Gap
% ## "Situate in research: What are the conclusions of existing literature for that problem / situation in this domain?"
% ## What is the problem or issue with that existing literature?
The literature on adaptation techniques is fragmented and FedNAS methods often lack clarity regarding the targeted FL setting and challenges addressed. As a result, extending and re-using existing techniques remains difficult. This poses a problem for FedNAS researchers, since they need to trade off which challenges to address for their targeted FL setting without a clear overview of adaptation techniques that would be useful for that setting. Prior literature surveys~\cite{fl_to_nas_survey_2021}~\cite{nas_hpo_fl_survey_2023}~\cite{multi-objective_methods_in_fl_2025} focus on FedNAS methods on the whole, but not the individual adaptation techniques they use. Prior surveys also only analyse a small fraction of the growing FedNAS literature. To aid in the development new FedNAS methods, we set out to answer our research question:

\vspace{1em}
\textbf{What challenges arise from particular FL settings for FedNAS methods and which adaptation techniques address them in the literature?}
\vspace{1em}

% # "Study: How investigate? Process, context & why?"
% ## "Indicate that this study addresses that problem or issue and state how."
% ## "Describe the study, sample, and method for addressing that problem or issue."
To tackle our research question, we conduct a systematic literature review of papers that present FedNAS methods. We employ grounded theory and the methodology from \cite{cdml_2024}. For our review we consider papers that make modifications to NAS methods motivated by the FL setting.

% ### from targeted FL settings and their challenges to adaptation techniques
We use a set of fine-grained parameters to characterise the targeted FL setting of each FedNAS method. Based on this characterisation, we identify the violated centralised NAS assumptions and catalog the challenges that arise from them. Next, we extract unrefined adaptation techniques from the FedNAS methods and iteratively refine and merge them to obtain a coherent set of mutually exclusive and collectively exhaustive adaptation techniques. We analyse how each adaptation technique works towards, against, or does not affect each FedNAS challenge, and present our findings in the form of a discussion for each adaptation technique and as an overview a table. 

% # "Conclusion"
% ## "Describe what you found. State explicitly how these findings extend and contribute to existing knowledge."
% - first sentence: what is the main goal
% - following sentences: contributions to knowledge and why they are relevant 
Our review aims to support the creation of new FedNAS methods by researchers. By identifying the source of challenges and elaborating them, we provide clarity on the expected challenges for a targeted FL setting. Based on the expected challenges, FedNAS researchers can use our overview of adaptation techniques to guide the design of new FedNAS methods and decide whether to re-use existing techniques, extend them or develop new ones.

% # Outline
In Chapter \ref{chapter:background}, we cover the background required for this thesis and related work. In Chapter \ref{chapter:method}, we describe the method with which we conduct our literature review and give an overview of the included literature. In Chapter \ref{chapter:fednas_challenges} we highlight the FedNAS challenges and how they arise from different FL settings. In Chapter \ref{chapter:adaptation_techniques}, we describe the adaptation techniques we found and how they overcome FedNAS challenges. In Chapter \ref{chapter:discussion}, we conduct a discussion about our work. Chapter \ref{chapter:conclusion} contains our conclusion.