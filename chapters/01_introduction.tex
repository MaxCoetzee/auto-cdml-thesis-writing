% !TeX root = ../main.tex
% Add the above to each chapter to make compiling the PDF easier in some editors.
\chapter{Introduction}\label{chapter:introduction}

% hook
%% hook a) "what is the domain and why is it important?"
Both Neural Architecture Search (NAS) and Federated Learning (FL) have made significant progress independently in the past decade, and both are increasingly adopted in practice. To benefit from the advantages of NAS methods in FL, researchers have started combining them by using NAS in the FL setting. 
%% => opening paragraph begs for
%%%      1. explanations of NAS and FL
%%%      2. why each is important
%%%      3. why applying NAS in FL is important 

%%% NAS explanation and why is it important
\textit{Neural Architecture Search} (NAS) automates the process of engineeering neural network architectures for Deep Learning application domains~\cite{nas_survey_2019}. This stands in contrast to the traditional, labourious approach to applying Deep Learning. Traditionally a team of domain experts and Deep Learning experts engineer a well-suited architecutre based on expert knowledge and trial-and-error. 

%%% second reason for why NAS is important
NAS does not only reduce manual effort, but can also be used to find architectures that perform better than architectures humans have designed for specific application domains~\cite{nasnet_2018}~\cite{amoebanet_2019}~\cite{mobilenetv3_2019}~\cite{efficientnetv2_2021}.

%%% FL explanation
\textit{Federated Learning} (FL) is a machine learning method whereby \textit{clients} collaboratively train a model without sharing their data. Weight updates to the shared model are coordinated by a central \textit{server}.

%%% why FL is important 
FL was invented by Google to enable privacy-preserving usage of the increasing volume of privacy-sensitive data stored on edge devices. Before FL was invented, it was typical for distributed privacy-sensitive data to either not be used at all for machine learning or it was collected in a central location for training — creating the risk of a major breach by malicious actors.  

%%% increasing relevance of FL 
Although FL was originally targeted at the usage of distributed data on edge devices, the usage of distributed data silos (at the organisational level) have become another use case. The former is referred to as the \textit{cross-device} FL setting and the latter as the \textit{cross-silo} FL setting. Both cross-device and cross-silo FL have since been adopted for various ML tasks in production systems by organisations like Google~\cite{gboard_fl_2018}, Apple~\cite{apple_fl_case_study_2025} and Owkin~\cite{owkin_fl_drug_discovery_in_prod_2022}.

%%% why NAS for FL is important: generic benefits and it is a good fit 
By using NAS in the FL setting, practicioners benefit from the generic benefits of NAS mentioned above as well as several benefits specific to the FL setting:
\begin{itemize}
    \item A large body of work in NAS focuses on finding smaller architectures with reduced inference latency that still have reasonable accuracy \cite{nas_1000_papers_2023}. Such lightweight architectures are ideal for deployment on the resource-constrained clients in the cross-device FL setting.
    \item \cite{fl_advances_and_open_problems_2021} note that predefined architectures may not be an optimal choice for FL. Since client data is not visible to model developers, a predefined architecture selected by model developers may contain components redundant for generalizing well from certain client data sets.
	\item Predefined architectures may perform poorly on another prevalent characteristic of the FL setting: data that is not independently and identically distributed (non-i.i.d.). % [TODO: why? and why is NAS better?]
\end{itemize}

% difference in assumptions creates challenges for adopting NAS in FL

%% hook b) "What is the overall problem or situation in that domain?"
%%% overall problem in FedNAS domain: traditional NAS is hard to use for FL
%%% TODO: justify usage of "However" or drop it 
However, Using NAS in the FL setting is not straightforward. Research on NAS methods has traditionally focused on a centralized setting as opposed to the distributed FL setting. This makes many NAS methods unfeasable for direct application in the FL setting, because NAS methods designed for the centralized setting can make several assumptions about the search process that do not hold in the FL setting (see Table 1). Instead, practicioners need to adapt NAS methods to the FL setting, giving rise to \textit{Federated Neural Architecture Search}~\cite{fednas_2021} (FedNAS) methods.

%%% get more specific about how NAS in FL is hard: assumption discreptancies create challenges
Assumptions that hold for NAS in the centralized setting, but do not hold for the FL setting, make it challenging to adapt NAS methods to the FL setting. Table 1 illustrates these discreptencies as well as the resulting challenges faced by FedNAS methods. 

% TODO: mark assumptions that also hold in cross-silo setting 
%\begin{table}[htbp]
%    \begin{tabular}{|c|c|c|}
%        Assumption in Centralized Setting & Reality in FL Setting & Resulting Class of Challenges \\
%        \csvreader[head to column names]{data/challenges.csv}{}%
%            {\centralizedassumptions & \flreality & \challenges}
%    \end{tabular}
%    \centering
%    \caption{Discreptencies between NAS in the centralised setting and the FL setting and the resulting challenges. \textit{Worker nodes} perform the search in the centralized setting. A \textit{server} and \textit{clients} perform the search in the FL setting.}
%    \label{table:assumptions}
%\end{table}

%%% why overcoming some challenges is mutually exclusive
Practicioners need to choose which set of challenges they are interested in addressing for their particular use case, since overcoming one challenge typically comes at the expense of others. For example, consider a particular client that might have a lot of data that could contribute towards the architecture search, but constantly drops out of communication rounds. Practicioners can choose to prioritize either a) avoiding delays due to stragglers or b) waiting for stragglers to ensure model fairness.

%%% practical problem: practicioners want to easily use FedNAS methods or build new FedNAS methods based on their needs
The possible use cases and subsets of challenges to overcome is large and consequentially a growing body of FedNAS methods have been created by practicioners. To this end an overview of FedNAS methods would be useful.
why:
- overview for researchers to view the landscape FedNAS methods
- overview for practicioners looking for existing FedNAS methods 
% why is it relevant that researchers develop new FedNAS methods?
%%% TODO: maybe add reason why a regular FedNAS user might create their own FedNAS method? Not only reasearcher do this
- overview for practicioners wanting to create a FedNAS method for their specific problem
However, it is not always clear to practicioners if a FedNAS method exists that well-suited for their particular use case and if one doesn't exist 
%%% why is literature review on adaptation techniques relevant?
%%%  another problem in FedNAS domain:- every problem is different => off-the-shelf solutions don't work => why?
%%% - need composable adaptation techniques to create fednas methods 
- scenarios for which a FedNAS method exists that suits a problem
Practicioners use existing FedNAS methods to solve a problem.
create new FedNAS methods, either to adapt new NAS methods to the FL setting or to adapt NAS methods in new ways to overcome different sets of challenge classes. To this end, it is useful to use existing literature on FedNAS methods and avoid re-inventing adaptation techniques for overcoming each of the challenge classes. A considerable amount of literature on FedNAS methods has appeared in recent years, resulting in a large number of novel adaptation techniques. Unfortunately, adaptation techniques are scattered throughout the increasingly large volume of FedNAS literature, putting a burden on researchers intersted in re-using them for new FedNAS methods. 
% why is fragemented literature a practical problem?

% Why challenge instead optimization objective? => optimization objective requires real-valued domain that can be optimized, but challenges can be yes/no or discrete

% Gap
%% Situate in research: What are the conclusions of existing literature for that problem / situation in this domain? 
There have been prior literature surveys on FedNAS methods~\cite{fl_to_nas_survey_2021}~\cite{nas_hpo_fl_survey_2023}~\cite{multi-objective_methods_in_fl_2025}. \cite{fl_to_nas_survey_2021} is an early survey that characterises FedNAS methods on the whole. The survey differentiates FedNAS methods into offline vs. online architecture search and single- vs. multi-objective methods. \cite{nas_hpo_fl_survey_2023} gives a brief overview of the FedNAS landscape at the time, highlighting the major contributions each FedNAs method has made. \cite{multi-objective_methods_in_fl_2025} provides an overview of how multi-objective optimizaiton can be integrated into FL in general and includes sections that discuss how this is done specifically for FedNAS methods. 

%% identify opportunity, explain relevance: What is the problem or issue with that existing literature?
%%% prior surveys are not exhaustive
Prior literature surveys only analyze a fraction of the FedNAS literature. \cite{fl_to_nas_survey_2021} and \cite{nas_hpo_fl_survey_2023} are limited by the small amount of FedNAS literature available at the time. The volume of proposed FedNAS methods has grown substantially since. \cite{multi-objective_methods_in_fl_2025} only analyzes FedNAS methods that make use of multi-objective optimization (MOO), thereby excluding a large share of the literature.

% we are interested in a survey that will allow users to easily create FedNAS methods as they need, not only summarize the field so far: usually the goal of the survey is to simply give an overview of the landscape, our review does that as well as provide an overview of the techniques available for practicioners to create their own FedNAS methods

% introduce adaptation techniques here, because the focus on them is part of the gap
To deal with the challenges faced by FedNAS methods, FedNAS practicioners employ what we shall call \textit{adaptation techniques} to adapt NAS methods to the FL setting. For example, naively using a supernet-based NAS method in the cross-device FL setting by having each client train the entire supernet, % "training the entire supernet" is ambiguous, need to reformulate
would result in detrimental completion times. This embodies the computational efficiency challenge, and one FedNAS method~\cite{fedoras_2022} overcomes it by \textit{adapting} the subnet sampling process of X NAS method, such that only subnets within the client's training budget get selected for training.

%%% why the focus on adaptation techniques and the challenge classes they overcome is useful
- users can pick FedNAS method for their use case
- researcher can create new, better FedNAS methods
Adaptation techniques are composable design patterns extracted from the FedNAS literature that make it easy to compose new FedNAS methods suited towards new tasks.

%%% prior surveys don't focus on adaptation techniques and challenge classes
None of the prior literature surveys provide a consolidated body of knowledge that can inform researchers on adaptation techniques used by FedNAS methods. \cite{fl_to_nas_survey_2021} and \cite{nas_hpo_fl_survey_2023} only analyse and summarise FedNAS methods on the whole. \cite{multi-objective_methods_in_fl_2025} analyses how MOO is used within FedNAS methods. The prior surveys do not identify individual adaptation techniques responsible for overcoming sets of challenge classes.


Extracting the adaptation techniques used by FedNAS methods and organising them into a single consolidated body of knowledge, would allow researchers to easily make use of this knowledge to compose new FedNAS methods tailored to overcoming a specific set of challenge classes relevant to them. This leads us to our research question:

% TODO: reformulate better
\vspace{1em}
(RQ) \textbf{How and which challenge classes do adaptation techniques described in the literature overcome?}
\vspace{1em}

% Study: How investigate? Process, context & why?
%% Indicate that this study addresses that problem or issue and state how.
%% Describe the study, sample, and method for addressing that problem or issue.
To answer our research question, we perform a systematic literature review of adaptation techniques used by FedNAS methods and their effects on overcoming challenge classes. We divide our approach into 5 steps:

\begin{enumerate}
    \item \textbf{Literature Selection:} We follow the guidelines and flow diagrams provided by PRISMA 2020 \cite{prisma_2020} for inclusion and exclusion of papers and perform forward and backwards citation searching. Each paper contains one or more FedNAS methods. 
    \item \textbf{Adaptation Technique Extraction:} Once the set of included papers is fixed, we analyse each paper individually, extracting the adaptation techniques it uses and summarising them.
    % how to measure conceptual similarity?
    \item \textbf{Merge Highly-Similar Adaptation Techniques:} We then merge conceptually highly-similar adaptation techniques into a single representative adaptation technique.
    \item \textbf{Categorise Adaptation Techniques:} After merging, we categorise the adaptation techniques based on conceptual similarity and deliver a taxonomy of adaptation techniques.
    \item \textbf{Map FL Challenge Types onto Adaptation Techniques:} Next, we discuss how each adaptation technique works towards, against, or does not affect overcoming each of the FL challenge classes and provide a table with an overview as an end result. 
\end{enumerate}
 
% Conclusion
% Describe what you found.
% State explicitly how these findings extend and contribute to existing knowledge.
Our review organizes the n extracted adaptation techniques into a single taxonomy that gives researchers an overview of the FedNAS landscape through the lens of adaptation techniques. Our discussions on each adaptation technique helps researchers find and choose adaptation techniques relevant to their problem. 

% Outline
In chapter 2 we cover the background required for this thesis and related work. In chapter 3 we describe the method with which we conduct our literature review in detail. In chapter 4 we explain our process of including FedNAS literature and give an overview of the included FedNAS literature. In chapter 5 we present our taxonomy of adaptation techniques and explain the effect of adaptation techniques on challenge classes. In Chapter 6 we conduct a discussion about our work. Chapter 7 contains our conclusion.
