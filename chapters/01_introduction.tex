% !TeX root = ../main.tex
% Add the above to each chapter to make compiling the PDF easier in some editors.
\chapter{Introduction}\label{chapter:introduction}

% # "hook"
% ## hook a) "what is the domain and why is it important?"
Both \textit{Neural Architecture Search} (NAS) and \textit{Federated Learning} (FL) have made significant progress independently in the past decade. To benefit from the advantages of NAS methods in FL, researchers have started combining them by using NAS in FL~\cite{fednas_2021}~\cite{rt_fed_evo_nas_2020}~\cite{spider_2021}~\cite{fedoras_2022}~\cite{finch_2024}~\cite{efnas_2024}. 
% => opening paragraph begs for
%      1. explanations of NAS and FL
%      2. why each is important
%      3. why applying NAS in FL is important 

% ### NAS explanation and why it is important
NAS automates the process of engineering neural network architectures for Deep Learning application domains~\cite{nas_survey_2019}. This saves manual labour compared to the traditional approach driven by a team of experts. Additionally, NAS can find architectures that perform better than architectures humans have designed for specific application domains~\cite{nasnet_2018}~\cite{amoebanet_2019}~\cite{mobilenetv3_2019}~\cite{efficientnetv2_2021}.

% ### FL explanation and why it is important
\textit{Federated Learning} (FL) is a machine learning method whereby \textit{clients} collaboratively train a model without sharing their local data. FL enables privacy-preserving machine learning on the increasing volume of privacy-sensitive, distributed data and avoids centrally collecting client data. FL has been adopted for various ML tasks in production systems by organisations such as Google~\cite{gboard_fl_2018}, Apple~\cite{apple_fl_case_study_2025}, and Owkin~\cite{owkin_fl_drug_discovery_in_prod_2022}.

% ## why NAS for FL is important: predefined architectures don't work well in FL 
% TODO: clearly indicate that the problems with predefined architectures come from the same root cause as the challenges for FedNAS methods (i.e. the FL general setting), but are not the same as the challenges with FedNAS
Using NAS in FL reduces manual effort for practitioners and potentially mitigates the limitations of predefined architectures. Predefined architectures are often suboptimal in FL, because client data is hidden and client data distributions as well as client hardware vary. As a result, predefined architectures may include redundant components, underfit some clients while overfitting others, and lead to a large variance in training times across clients. NAS offers a promising way to address these issues by tailoring architectures to the specific clients and their data.

% ## hook b) "What is the overall problem or situation in that domain?"
% ### overall problem in FedNAS domain: assumption discrepancies create challenges for using NAS in FL
However, NAS research has focused on a centralised setting as opposed to the distributed FL setting. This makes many NAS methods infeasible for direct application in FL, because centralised NAS can make several assumptions about the search process that do not hold in the FL setting. Each assumption discrepancy creates a \textit{challenge} for using NAS in FL and researchers have developed \textit{adaptation techniques} for overcoming them. Applying adaptation techniques to NAS methods results in \textit{Federated Neural Architecture Search}~\cite{fednas_2021} (FedNAS) methods.
% ### I am aware that starting paragraphs with "However," is generally bad style, but I think in this case it nicely hooks up with the previous paragraph.

% ### challenge example: Limited Computational Resources
% TODO: list 2-3 common challenges, but only give one fully fledged example
For example, one challenge arises from the fact that centralised NAS can assume worker nodes are computationally powerful, whereas clients in most FL settings are not. Since most centralised NAS methods place large computational burdens on worker nodes, practitioners using these NAS methods in FL without modification would experience detrimental search completion times. Researchers have developed adaptation techniques for reducing the computational burden on individual clients in FedNAS methods [TODO: cite]. This typically involves reducing the overall computational work and splitting it up into smaller units, which manifests as a challenge, because the implementation of the resulting FedNAS method tends to be complex.

% ### the relevant challenge subset depends on the targeted FL setting, which varies a lot
The subset of challenges faced by FedNAS methods depend on the specific FL setting. FL settings can differ in many aspects, such as the degree of client hardware heterogeneity, the number of participating clients, the degree of data volume imbalance between clients, etc. \cite{fl_advances_and_open_problems_2021}. Each FL setting violates centralised NAS assumptions to a different extent, making some challenges more relevant than others for that setting. For example, two major classes of FL settings appear in the literature: the \textit{cross-device} setting class, wherein clients are edge devices, and the \textit{cross-silo} setting class wherein clients are entire organisations. For FL settings in the cross-silo setting class, clients can be expected to be equipped with GPUs, making the challenge described above less relevant.

% ##### why overcoming some challenges can be mutually exclusive
FedNAS researchers need to prioritise which challenges to address for their targeted FL setting: the relevance of a challenge depends on how strongly the corresponding centralised NAS assumption is violated, and overcoming one challenge typically comes at the expense of neglecting others. For example, in a FL setting where some clients hold more informative data but frequently drop out, ignoring stragglers addresses the challenge caused by client unreliability and lowers search completion time but risks biasing the search towards always-available clients. Waiting for stragglers instead, addresses the challenge caused by varying client availability and improves fairness at the cost of slower and potentially less stable search.

% # Gap
% ## "Situate in research: What are the conclusions of existing literature for that problem / situation in this domain?"
Researchers have already conducted several literature surveys on FedNAS methods~\cite{fl_to_nas_survey_2021}~\cite{nas_hpo_fl_survey_2023}~\cite{multi-objective_methods_in_fl_2025}, but none are exhaustive and identify adaptation techniques. \cite{fl_to_nas_survey_2021} and \cite{nas_hpo_fl_survey_2023} characterise FedNAS methods on the whole, whereas \cite{multi-objective_methods_in_fl_2025} describes how multi-objective optimisation can be integrated into FL in general and includes sections that discuss how this is done specifically for FedNAS methods, but no survey focuses on individual adaptation techniques. \cite{fl_to_nas_survey_2021} and \cite{nas_hpo_fl_survey_2023} are limited by the small amount of FedNAS literature available at the time and \cite{multi-objective_methods_in_fl_2025} only analyses a small sample of FedNAS methods, since their focus is on FL in general.

% ## What is the problem or issue with that existing literature?
Researchers are developing an increasing amount of adaptation techniques for FedNAS methods, but extending and re-using existing techniques remains difficult, as the literature is fragmented and often lacks clarity regarding the targeted FL setting and challenges addressed. To ease the development new FedNAS methods, we set out to answer our research question:

\vspace{1em}
\textbf{How do adaptation techniques described in the literature address challenges of particular FL settings?}
\vspace{1em}

% # "Study: How investigate? Process, context & why?"
% ## "Indicate that this study addresses that problem or issue and state how."
% ## "Describe the study, sample, and method for addressing that problem or issue."
To answer our research question, we conduct a systematic literature review using grounded theory and lean on the methodology employed by \cite{cdml_2024}.  

We first identify papers that present FedNAS methods. Then, we perform open coding on each FedNAS method to extract unrefined adaptation techniques. Next, we perform axial coding by iteratively refining and merging unrefined adaptation techniques to obtain a coherent set of mutually exclusive and collectively exhaustive adaptation techniques. 
% ## I left out how the literature is selected, since it is not relevant to the intro, but will be described in detail in the method section. 

We then analyse how each adaptation technique works towards, against, or does not affect each FedNAS challenge, and aggregate these coded effects into two overview tables: one for FedNAS methods and one for adaptation techniques. These tables help practitioners decide on FedNAS methods suited to their use case or adaptation techniques that practitioners can use to create new FedNAS methods suited to their use case.
 
% # "Conclusion"
% ## "Describe what you found. State explicitly how these findings extend and contribute to existing knowledge."
Our review organises the n extracted adaptation techniques into a consolidated body of knowledge that provides FedNAS practitioners with an overview of the FedNAS landscape through the lens of adaptation techniques. Compared to existing surveys, our review is exhaustive of the FedNAS landscape published up to 2025. Additionally, our review of each adaptation technique and our overview tables help practitioners find and choose FedNAS methods relevant to their use case or construct new FedNAS methods by re-using appropriate adaptation techniques. 

% # Outline
In Chapter \ref{chapter:background}, we cover the background required for this thesis and related work. In Chapter \ref{chapter:method}, we describe the method with which we conduct our literature review in detail. In Chapter \ref{chapter:reviewed_literature}, we describe our process for including FedNAS literature and provide an overview of the reviewed FedNAS literature. In Chapter \ref{chapter:adaptation_techniques}, we present our taxonomy of adaptation techniques and our review of how adaption techniques overcome FedNAS challenges. In Chapter \ref{chapter:discussion}, we conduct a discussion about our work. Chapter \ref{chapter:conclusion} contains our conclusion.