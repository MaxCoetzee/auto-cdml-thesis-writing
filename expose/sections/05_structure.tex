\section{Structure}

% Outline: Was willst du in den einzelnen (Sub-)kapiteln behandeln? Welche Frage beantwortest du?

% Please provide a first outline of your thesis. The number of sections in a chapter and the level of detail should correspond to the importance of the sections. A 1st and 2nd level outline are sufficient for the proposal. The outline must be annotated. This means that the content and aim of each chapter must be described in the form of questions. The estimated number of pages per section must be added to the comments. Example: 1. Introduction (2 pages) 1.1 Problem definition (1 page) || Comment... ..

\begin{enumerate}
    \item \textbf{Introduction} \textit{(3 pages)}
    
    What is the motivation behind this thesis? What is the relevance of this thesis? What problem does this thesis try to solve? What kind of approach does this thesis take to solve that problem? How is the thesis structured?
    
    \item \textbf{Background} \textit{(6 pages)}
    
    What background knowledge is required to understand this thesis?  
    \begin{enumerate}
        \item Neural Architecture Search (2.5 pages) 

        What is NAS? What is it used for? What is a NAS method? What are the origins of NAS? What environment are NAS methods typically developed in?

        \item Federated Learning (2.5 pages) 

        What is Federated

        \item Federated Neural Architecture Search (0.5 pages) 
        
        How do FedNAS methods relate to NAS and FL? What are the origins FedNAS methods and how have they been developed over the last couple of years?
    \end{enumerate}
    
    \item \textbf{Method} \textit{(5 pages)}

    How does this thesis achieve the objectives described in \autoref{sec:objectives}?
    \begin{enumerate}
        \item Method and Literature Selection (3 pages)
        \item Reviewed Literature (2 pages)
    \end{enumerate}

    \item \textbf{Challenges with Adapting Centralised NAS methods to FL} (10 pages): 
    \begin{enumerate}
        \item Relevant FL system parameters \textit{(3 pages)}
        
        What makes a FL system parameter relevant to adaptation techniques described in this thesis? Which FL system parameters are relevant and which ones are not?

        \item Assumption Discrepancies between NAS and FedNAS \textit{(3 pages)}
        
        What assumptions can be made in
        \item Adaptation Challenges (4 pages) How do FL system parameters influence challenges faced for adapting centralised NAS methods to  ?
    \end{enumerate}
    
    \item \textbf{Adaptation Techniques} (25 pages) What adaptation techniques are described in the literature? What can we learn from these adaptation techniques that we can use for adapting NAS methods to FL in the future? For which FL system parameter configuration was each adaptation technique developed? How do adaptation techniques work towards, against, or do not affect overcoming each of the challenges relevant to that FL system parameter configuration?
    \begin{enumerate}
        \item Adaptation Technique 1 \textit{(1 page)} 
        
        Description of the adaptation technique, the targeted FL system parameter configuration, the FedNAS methods that use it and a discusson on how it overcomes challenges.
        \item Adaptation Technique 2 \textit{(1 page)}
        \item ...
        \item Adaptation Technique 20 \textit{(1 page)}
        \item Overview \textit{(5 pages)} A large overview table that makes it easy to see which adaptation techniques are beneficial for overcoming which challenges at a glance.
    \end{enumerate}

    \item \textbf{Discussion} \textit{(2 pages)}

    \item \textbf{Conclusion} \textit{(1 page)}
\end{enumerate}

Totalling roughly \textit{... pages}