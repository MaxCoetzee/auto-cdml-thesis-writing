\section{Research Approach}

\subsection{Data Collection}

We obtain the set of literature relevant to our review by searching the abstract, title and keywords of literature in Scopus~\cite{scopus} with the search string "federated learning neural architecture search" and include literature that presents a FedNAS method that uses a centralised NAS method explicitly modified for FL. We exclude literature that (i) designs FedNAS methods from scratch, (ii) performs only hyperparameter optimisation, or (iii) does not provide sufficient methodological detail to extract adaptation techniques.

We extend this initial set of literature by recursively adding literature from the references of the literature that meets our inclusion and exclusion criteria until we obtain a set of literature for which no new literature can be added. 

\subsection{Data Analysis}

We employ a qualitative research approach, specifically a systematic literature review. In our review, we develop a conceptual model that links violated centralised NAS assumptions to resulting challenges for adapting a centralised NAS method to FL, and the adaptation techniques used to overcome these challenges as follows:

\begin{enumerate}
    \item \textbf{Challenges arising from Assumption Discrepancies:} We first define challenges that arise from using centralised NAS methods in FL, because of assumptions that hold for centralised NAS, but not for FL.
    \item \textbf{Influence of FL System Parameters on Challenges:} We then classify the influence of FL system parameters on the relevance of challenges as either "increasing" or "decreasing" the relevance of each challenge.
    \item \textbf{Adaptation Technique Conceptualisation:} Next, we extract adaptation techniques from the literature in two steps.
    \begin{enumerate}
        \item First, we perform open coding on each FedNAS method to extract unrefined adaptation techniques. Any modification to a NAS method that is explicitly motivated by the federated setting is initially coded as one unrefined adaptation technique.
        \item Then, we iteratively refine and merge the unrefined adaptation techniques (similar to ~\cite{cdml_2024}) to obtain a coherent set of adaptation techniques. Unrefined adaptation techniques with conceptually highly similar mechanisms are merged into a single representative adaptation technique.
    \end{enumerate}
    \item \textbf{Discuss FedNAS Challenges for Adaptation Techniques:} Afterwards, we discuss how each adaptation technique works towards, against, or does not affect overcoming each of the FedNAS challenges.
    \item \textbf{Table Overview:} Finally, we create a table that contains a coded vector of effects over the challenges for each adaptation technique based on the prior discussion. 
\end{enumerate}

Practitioners can use our classification from 2. to first identify challenges relevant to their targeted FL system parameters and then refer to our table from 5. to gain an overview at a glance of which adaptation techniques could be useful to them.