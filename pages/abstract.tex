\chapter{\abstractname}

Applying techniques from Neural Architecture Search (NAS) to Federated Learning (FL) has been fruitful (remove) in recent years. The combination was identified as a promising research (remove) direction by ~\cite{fl_advances_and_open_problems_2021}. It has yielded methods for finding architectures that deal with the challenges imposed by the FL setting.

% start with a problem that the user can identify them with: write to Nick and Niclas
% explain why its interesting, what are great things you could do

% transfer progress from nas to fl
% motivation for nas applied to fl -> show conflict
% nas methods for fl a lot, different -> for certain settings work well, but others not
% design of nas for fl implicated design
% research problem: core components of nas methods of fl not described yet
% surveys for exist, but literatue is very spread out and not holistic
% suitability for NAS methods not clear
% practical problem: specialized NAS methods for FL methods, how to decide which one to use?

% Wissen das fehlt, wie entscheidet man was man anwendet
% Was für Wissen brauchen wir

% Don't write: No paper exists, nobody knows, rather frame positively and motivate

% Introduction Structure:
% Context: NAS is cool, we want to use it in FL, but isn't transferrable
% Practical Problem: Can't decide NAS methods for FL -> we want to be able to do that
% Research Problem: Explain why current state of research 
% What needs to be done

Research into NAS has grown rapidly ~\cite{nas_1000_papers_2023} since it was popularized by ~\cite{nas_seminal_2017}; consequently, literature on its application to FL has grown. The last survey on NAS applied to FL compared approaches of four papers ~\cite{fl_to_nas_survey_2021}. Since then, we have identified approximately 50 new papers. This motivates a new systematic survey of the landscape to identify progress and gaps in the literature. 

In this thesis, we propose a map of the literature landscape based on the FL challenges they address. We achieve this by systematically evaluating the literature and identifying which challenge it solves. 

We refer to the FL challenges described in ~\cite{fl_seminal_2017}, i.e., non-IID data, limited communication, client heterogeneity, privacy of client data, and break them down into smaller subchallenges — each subchallenge being associated with a pattern in the literature. We include personalized FL ~\cite{personalized_fl_2023} as an additional subchallenge that was not originally posited, but has since drawn the community's attention. 

We then analyze how the subchallenges are addressed and focus on the contribution of the used NAS method towards overcoming the subchallenge. For each subchallenge, we keep track of the NAS types used (following ~\cite{nas_1000_papers_2023}, ~\cite{systematic_nas_survey_2024}) and assess whether the underexplored methods are candidates for future research.
